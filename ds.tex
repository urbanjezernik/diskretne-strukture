\documentclass[11pt]{book}
\usepackage[slovene]{babel}
\usepackage[utf8]{inputenc}
\usepackage{amsmath}
\usepackage[dvipsnames]{xcolor}
\usepackage{titlesec}
% \usepackage{amssymb}
\usepackage{pstricks,pst-plot,pst-math}
\usepackage{pstricks-add}
\usepackage{graphicx}
\usepackage{enumerate}
\usepackage{color}
\usepackage{fouriernc}
\usepackage{microtype}
\usepackage{MnSymbol}
\usepackage{tikz-cd}
\usetikzlibrary{backgrounds}
\usepackage{wrapfig}
\usepackage{geometry}
\geometry{
    a4paper,
    left=45mm,
    right=45mm,
    top=20mm,
    bottom=20mm
    }
    \usepackage{comment}
    

\usepackage{amsthm}

\usepackage{changepage}   % for the adjustwidth environment
\usepackage{hyperref}
\hypersetup{
    colorlinks=true,
    linkcolor=cyan,
    filecolor=magenta,      
    urlcolor=cyan
    }

\usepackage[backgroundcolor=svetlosiva,linecolor=siva,textsize=footnotesize]{todonotes}

\pagestyle{plain}

\usepackage{enumitem}
\setlist[description]{leftmargin=\parindent,labelindent=\parindent, font=\normalfont\itshape\textbullet\space}


\def\NN{\mathbf{N}}
\def\ZZ{\mathbf{Z}}
\def\QQ{\mathbf{Q}}
\def\RR{\mathbf{R}}
\def\CC{\mathbf{C}}
\def\conclass{\mathcal{C}}
\def\11{\mathbf{1}}
\def\FF{\mathbf{F}}
\def\Fcal{\mathcal{F}}
\def\EE{\mathbf{E}}
\def\PP{\mathbf{P}}
\def\HH{\mathbf{H}}
\def\youngsym{\sigma_{\lambda}}

\DeclareMathOperator\image{im}
\DeclareMathOperator\sgn{sgn}
\DeclareMathOperator\Res{Res}
\DeclareMathOperator\Ind{Ind}
\DeclareMathOperator\Rep{Rep}
\DeclareMathOperator\mult{mult}
\DeclareMathOperator\Izotip{Izotip}
\DeclareMathOperator\MK{MK}
\DeclareMathOperator\tr{tr}
\DeclareMathOperator\Irr{Irr}
\DeclareMathOperator\SU{SU}
\DeclareMathOperator\characteristic{char}
\DeclareMathOperator\kk{k}
\DeclareMathOperator\cl{cl}
\def\GAP{\texttt{GAP}}
\DeclareMathOperator\inv{inv}
\DeclareMathOperator\Eigenvalues{Spec}
\DeclareMathOperator\Eigenspace{ES}
\DeclareMathOperator\fun{fun}
\DeclareMathOperator\HS{HS}
\DeclareMathOperator\St{St}
\DeclareMathOperator\Realpart{Re}

\DeclareMathOperator\DNO{DNO}
\DeclareMathOperator\KNO{KNO}


\DeclareMathOperator\Aut{Aut}
\DeclareMathOperator\GL{GL}
\DeclareMathOperator\glfrak{\mathfrak{gl}}
\DeclareMathOperator\slfrak{\mathfrak{sl}}
\DeclareMathOperator\U{U}
\DeclareMathOperator\SL{SL}
\DeclareMathOperator\PSL{PSL}
\DeclareMathOperator\SO{SO}
\DeclareMathOperator\Gal{Gal}
\DeclareMathOperator\Sym{Sym}
\DeclareMathOperator\Homeo{Homeo}
\DeclareMathOperator\Cay{Cay}
\DeclareMathOperator\Isom{Isom}
\DeclareMathOperator\id{id}
\DeclareMathOperator\supp{supp}
\DeclareMathOperator\End{End}
\DeclareMathOperator\Mat{Mat}
\DeclareMathOperator\Cone{Cone}
\DeclareMathOperator\diam{diam}
\DeclareMathOperator\Ad{Ad}
\DeclareMathOperator\imaginary{Im}

\def\definicija{\color{rdeca}\bf\em}
\def\vprasanje{\color{oranzna}}
\def\literatura{\color{modra}}
\def\vaje{{\literatura ($\to$ vaje)}}
\def\kljuka{$\checkmark$}

\theoremstyle{definition}

\newtheoremstyle{zgled}
 {}{}%
 {\color{zelena}}
 {}%
 {\color{zelena}\bfseries}%
 {\color{zelena}.}%
 { }{}

\theoremstyle{zgled}
\newtheorem*{zgled}{Zgled}

\newtheoremstyle{odprtproblem}
 {}{}%
 {\color{oranzna}}
 {}%
 {\color{oranzna}\bfseries}%
 {\color{oranzna}.}%
 { }{}

\theoremstyle{odprtproblem}
\newtheorem*{odprtproblem}{Odprt problem}

\newtheoremstyle{domacanaloga}
 {}{}%
 {\color{vijolicna}}
 {}%
 {\color{vijolicna}\bfseries}%
 {\color{vijolicna}.}%
 { }{}

\theoremstyle{domacanaloga}
\newtheorem*{domacanaloga}{Domača naloga}

\newenvironment{dokaz}
    {\color{siva}\begin{proof}}
    {\end{proof}}

\newtheoremstyle{izrek}
 {}{}% above, below 
 {\color{black}\itshape}
 {}% indent
 {\color{black}\bfseries}%
 {\color{black}.}%
 { }{}

\theoremstyle{izrek}
\newtheorem*{izrek}{Izrek}

\newtheorem*{trditev}{Trditev}
\newtheorem*{pomoznatrditev}{Pomožna trditev}

\newtheorem*{lema}{Lema}

\newtheorem*{posledica}{Posledica}

\newenvironment{povzetek}
    {
\smallskip
\begin{center}
\color{svetlosiva}
\begin{tabular}{|p{0.7\textwidth}}
    }
    {
\end{tabular}
\end{center}
\smallskip
    }


\definecolor{rdeca}{rgb}{0.62, 0.16, 0.10}
\definecolor{zelena}{rgb}{0.15, 0.4, 0.20}
\definecolor{oranzna}{rgb}{0.72, 0.38, 0.082}
\definecolor{rjava}{rgb}{0.7490196078431373, 0.3686274509803922, 0.1843137254901961}
\definecolor{modra}{rgb}{0.2784313725490196, 0.5411764705882353, 0.8392156862745098}
\definecolor{vijolicna}{rgb}{0.48627450980392156, 0.2980392156862745, 0.792156862745098}
\definecolor{siva}{rgb}{0.5, 0.5, 0.5}
\definecolor{svetlosiva}{rgb}{0.7, 0.7, 0.7}


\titleformat{\section}
  {\color{rdeca}\LARGE\bf}{\thesection}{1em}{}
\renewcommand{\thesubsection}{}
\titleformat{\subsection}
  {\Large\bf}{}{1em}{}

\title{\bf Diskretne stukture}
\author{Urban Jezernik}

% za generiranje html dokumenta s stilom mystyle.css uporabi:
% pandoc ds.tex --toc --toc-depth=2 --metadata date="`date -u "+%d. %m. %Y"`" --template template.html -c mystyle.css -s --mathjax -o index.html


\begin{document}

\baselineskip=14pt

\maketitle

\setcounter{tocdepth}{1}
\tableofcontents

\newpage

\subsection*{Kratek opis predmeta}

Pri predmetu se bomo najprej naučili, kaj točno so \emph{izjave} in kako jih matematično \emph{formalizirati}. Eden pomembnih ciljev tega je eksakten opis  \emph{sklepanja}, ki ga uporabljamo počez matematike. Te koncepte bomo najprej razvili v osnovnem \emph{izjavnem računu}, nato pa ga bomo še posplošili do \emph{predikatnega računa}, s katerim bomo podrobneje raziskali matematične izjave. 

\begin{zgled}
Nepridipravi so razbili vhodna vrata FMF. Glavni osumljenci so študenti Ana, Bor in Cveto. Ko jih vprašamo, kdo je kriv, odgovorijo z naslednjimi izjavami:

\begin{itemize}
    \item Ana: ``Bor je kriv, Cveto pa ne.''
    \item Bor: ``Če je kriva Ana, je kriv tudi Cveto.''
    \item Cveto: ``Jaz nisem kriv, toda vsaj eden od drugih dveh je kriv.''
\end{itemize}

Pri predmetu se bomo naučili, kako lahko na \emph{sistematičen način} odkrijemo, kdo je lagal, če krivi lažejo, nedolžni pa govorijo resnico.
\end{zgled}

Za tem bomo pri predmetu spoznali nekaj osnovnih diskretnih struktur, po katerih se ta predmet imenuje. Najprej bomo raziskali \emph{relacije}, s katerimi opisujemo odnose med elementi dane množice. Pomemben poseben primer teh so \emph{urejenosti}, ki posplošujejo običajne ureditve števil po velikosti. Najbolj podrobno si bomo ogledali \emph{grafe}, s katerimi lahko abstraktno predstavimo mnogo pomembnih primerov relacij.

\begin{zgled}
Graf je diskretna struktura, pri kateri dano množico \emph{vozlišč} povežemo s \emph{povezavami}. Z grafi lahko opišemo veliko različnih vrst omrežij, na primer internetno omrežje, omrežje prijateljskih povezav na Facebooku, omrežja veriženja blokov (kriptovalute) \dots 

Konkreten graf na spodnji sliki se imenuje {\definicija Petersenov graf}. Ta graf bomo tekom predmeta večkrat srečali. Na koncu predmeta bomo znali \emph{dokazati}, da se tega grafa ne da narisati v ravnini, brez da bi se vsaj dve povezavi sekali.

\begin{figure}[h]
    \centering
    \includegraphics[width=0.5\linewidth]{img/opis-petersen.png}
    \caption{Petersenov graf}
\end{figure}
\end{zgled}

\newpage

\subsection*{Literatura}

\begin{itemize}
\item {\literatura G. Fijavž, \href{http://matematika.fri.uni-lj.si/ds/ds.pdf}{\emph{Diskretne strukture}}, elektronska knjiga, 2015.} 
\item {\literatura M. Juvan in P. Potočnik, \href{http://www.dmfa-zaloznistvo.si/ipmr/1662.htm}{\emph{Teorija grafov in kombinatorika}}, DMFA-založništvo, Ljubljana 2000.}
\item {\literatura N. Prijatelj, \emph{Osnove matematične logike I}, DMFA-založništvo, Ljubljana, 1992.}
\end{itemize}

\todo{Dodaj literaturo za vaje.}

\chapter{Izjavni račun}

V tem poglavju si bomo pogledali, kako \emph{formaliziramo} preproste izjave in kako \emph{dokazujemo} njihovo veljavnost oziroma neveljavnost.

\section{Izjave in izjavni vezniki}

{\definicija Izjava} je poved, ki je bodisi resnična bodisi lažna.

\begin{zgled} \leavevmode
\begin{itemize}
    \item Ena in ena je tri. \emph{Lažna izjava.}
    \item Ena in ena je dve. \emph{Resnična izjava.}
    \item Koliko je ena in ena? \emph{Ni izjava.}
    \item Pojdimo na kavo! \emph{Ni izjava.}
\end{itemize}
\end{zgled}

Izjave lahko razdelimo na dve skupini \emph{po vsebini}, in sicer:
\begin{itemize}
    \item {\definicija resnične izjave}, ki imajo resničnostno vrednost $1$ ali $\top$ ali \texttt{true},
    \item {\definicija lažne izjave}, ki imajo resničnostno vrednost $0$ ali $\bot$ ali \texttt{false}.
\end{itemize}
Po \emph{zgradbi} oziroma \emph{obliki} pa izjave razdelimo na:
\begin{itemize}
    \item {\definicija osnovne}, ki ne vsebujejo izjavnih veznikov,
    \item {\definicija sestavljene}, ki vsebujejo izjavne veznike.
\end{itemize}

\begin{zgled} \leavevmode
\begin{itemize}
    \item Vreme je lepo. \emph{Osnovna izjava.}
    \item Špela gre v hribe. \emph{Osnovna izjava.}
    \item Vreme je lepo \emph{in} Špela gre v hribe. \emph{Sestavljena izjava.}
    \item \emph{Če} je vreme lepo, \emph{potem} gre Špela v hribe. \emph{Sestavljena izjava.}
    \item Špela \emph{ne} gre v hribe. \emph{Sestavljena izjava.}
\end{itemize}
\end{zgled}

Naj bo $n \in \NN_0$. {\definicija Izjavni veznik reda $n$} (ali {\definicija $n$-mestni izjavni veznik}) je funkcija, ki vsaki urejeni $n$-terici ničel in enic priredi vrednost $0$ ali $1$.

\begin{zgled} \leavevmode
\begin{itemize}
    \item Primer izjavnega veznika reda $1$ je {\definicija negacija}. Simbol za ta veznik je $\lnot$. Če je $p$ izjava, njeno negacijo označimo kot $\lnot p$ in preberemo kot \emph{ne $p$} ali kot \emph{ni res, da velja $p$}. Negacija $1$-terici $0$ priredi vrednost $1$, $1$-terici $1$ pa priredi vrednost $0$.
    
    \begin{table}[h]
        \centering
        \begin{tabular}{c|c}
            $p$ & $\lnot p$ \\ \hline
            0 & 1 \\
            1 & 0
        \end{tabular}
        \caption{Resničnostna tabela negacije}
    \end{table}

    \item Oglejmo si nekaj pomembnih dvomestnih izjavnih veznikov. Njihovi predpisi so zbrani v tabeli.
    \begin{itemize}
        \item {\definicija Konjunkcija} izjav $p$ in $q$ ima simbol $p \land q$, kar preberemo kot \emph{$p$ in $q$}.
        \item {\definicija Disjunkcija} izjav $p$ in $q$ ima simbol $p \lor q$, kar preberemo kot \emph{$p$ ali $q$}.
        \item {\definicija Implikacija} izjav $p$ in $q$ ima simbol $p \Rightarrow q$, kar preberemo kot \emph{če $p$, potem $q$} ali kot \emph{iz $p$ sledi $q$} ali kot \emph{$p$ je zadosten pogoj za $q$} ali \emph{kot $q$ je potreben pogoj za $p$}.
        \item {\definicija Ekvivalenca} izjav $p$ in $q$ ima simbol $p \Leftrightarrow q$, kar preberemo kot \emph{$p$ natanko tedaj, ko $q$} ali kot \emph{$p$, če in samo če $q$} ali kot \emph{$p$ je ekvivalentno $q$}.
    \end{itemize}

    \begin{table}[h]
        \centering
        \begin{tabular}{cc|cccc}
            $p$ & $q$ & $p \land q$ & $p \lor q$ & $p \Rightarrow q$ & $p \Leftrightarrow q$ \\ \hline
            1 & 1 & 1 & 1 & 1 & 1 \\
            1 & 0 & 0 & 1 & 0 & 0 \\
            0 & 1 & 0 & 1 & 1 & 0 \\
            0 & 0 & 0 & 0 & 1 & 1
        \end{tabular}
        \caption{Resničnostna tabela nekaterih pomembnih dvomestnih izjavnih veznikov}
    \end{table}

    \item Izjavni veznik reda $0$ je funkcija iz množice urejenih $0$-teric ničel in enic. Obstaja natanko ena taka $0$-terica, in sicer \emph{prazna $0$-terica}. Izjavni veznik reda $0$ je torej natanko določen s sliko te $0$-terice, za kar imamo dve možnosti, $0$ ali $1$.
    \begin{itemize}
        \item Izjavni veznik reda $0$, ki ima vselej resničnostno vrednost $0$, imenujemo {\definicija 0} in preberemo kot \emph{lažna izjava}.
        \item Izjavni veznik reda $0$, ki ima vselej resničnostno vrednost $1$, imenujemo {\definicija 1} in preberemo kot \emph{resnična izjava}.
    \end{itemize}
    Izjavnima veznikoma reda $0$ pravimo tudi {\definicija izjavni konstanti}.
\end{itemize}
\end{zgled}

Glede na zgornjo obravnavamo izjavnih veznikov redov $0$, $1$ in $2$ se lahko vprašamo, koliko je vseh $n$-mestnih izjavnih veznikov za poljuben $n \in \NN_0$. Vsak tak veznik je enolično določen s svojo resničnostno tabelo, v kateri zabeležimo vrednosti veznika v {\definicija izjavnih spremenljivkah} $p_1, p_2, \dots, p_n$, kjer vsak $p_i$ zavzema vrednosti $0$ ali $1$.

\begin{table}[h]
    \centering
    \begin{tabular}{cccc|c}
        $p_1$ & $p_1$ & $\cdots$ & $p_n$ & veznik($p_1$, $p_2$, \dots, $p_n$) \\ \hline
        1 & 1 & $\cdots$ & 1 & 0 ali 1 \emph{(dve možnosti)} \\
        1 & 1 & $\cdots$ & 0 & 0 ali 1 \emph{(dve možnosti)} \\
        $\vdots$ & $\vdots$ & & $\vdots$ & $\vdots$ \\
        0 & 0 & $\cdots$ & 0 & 0 ali 1 \emph{(dve možnosti)} \\
    \end{tabular}
    \caption{Resničnostna tabela $n$-mestnega izjavnega veznika}
\end{table}

Število vseh $n$-mestnih izjavnih veznikov torej izračunamo tako, da preštejemo število vseh možnih resničnostnih tabel. Število vrstic tabele je enako $2^n$, število vseh tabel pa je zato enako $2^{2^n}$.

\begin{table}[h]
    \centering
    \begin{tabular}{c|cc}
        $n$ & $2^n$ & $2^{2^n}$ \\ \hline
        0 & 1 & 2 \\
        1 & 2 & 4 \\
        2 & 4 & 16 \\
        3 & 8 & 256 \\
        4 & 16 & 65536 \\
        5 & 32 & $\sim 4 \cdot 10^9$ \\
        6 & 64 & $\sim 2 \cdot 10^{19}$
    \end{tabular}
    \caption{Hitra rast števila $n$-mestnih izjavnih veznikov}
\end{table}

\section{Izjavni izrazi}

{\definicija Izjavni izraz} definiramo induktivno na naslednji način:
\begin{itemize}
    \item \emph{Osnovni izraz 1:} Vsaka \emph{izjavna konstanta} (torej $0$ ali $1$) je izjavni izraz. 
    \item \emph{Osnovni izraz 2:} Vsaka \emph{izjavna spremenljivka} $p_1, p_2, \dots$ je izjavni izraz.
    \item \emph{Sestavljeni izraz:} Če je $f$ \emph{izjavni veznik} reda $n$ in so $A_1, A_2, \dots, A_n$ izjavni izrazi, potem je $(f(A_1,A_2,\dots,A_n))$ izjavni izraz.
\end{itemize}

Izjavni izrazi so torej izjave, ki jih dobimo iz $0,1$ in izjavnih spremenljivk z (večkratno) uporabo izjavnih veznikov.

\begin{zgled}
    Naj bodo $p,q,r$ izjavne spremenljivke. Tvorimo lahko izjavne izraze $(p \Rightarrow q)$, $(\lnot r)$, $((p \Rightarrow q) \land (\lnot r))$, \dots
\end{zgled}

Pri pisanju bolj zakompliciranih izjavnih izrazov se prične pojavljati mnogo oklepajev. V izogib pisanju prevelikega števila teh oklepajev uporabljamo naslednji {\definicija dogovor o prednostnem vrstnem redu veznikov}:
\begin{itemize}
    \item $\lnot$ ima prednost pred dvomestnimi vezniki,
    \item veznik iz $(\land, \lor, \Rightarrow, \Leftrightarrow)$ ima prednost pred vezniki desno od sebe,
    \item če isti veznik nastopi večkrat zapored, ima levi nastop prednost pred desnim,
    \item zunanji oklepaj spuščamo.
\end{itemize}

\begin{zgled} \leavevmode
\begin{itemize}
    \item Izjavni izraz $(p \Rightarrow (q \land r))$ pišemo krajše kot $p \Rightarrow q \land r$.
    \item Izraz $p \Rightarrow q \Rightarrow r \Rightarrow s$ je okrajšava za izjavni izraz $(((p \Rightarrow q) \Rightarrow r) \Rightarrow s)$.
    \item Izraz $p \lor \lnot q \Leftrightarrow r \Rightarrow q$ je okrajšava za izjavni izraz $(( p \lor (\lnot q)) \Leftrightarrow (r \Rightarrow p))$.
\end{itemize}
\end{zgled}

Izjavni izrazi vsebujejo izjavne spremenljivke, zato določajo neko resničnostno tabelo in s tem tudi nek izjavni veznik.

\begin{zgled}
Naj bodo $p,q,r$ izjavne spremenljivke in naj bo $f(p,q,r) = (p \Rightarrow q) \land \lnot r$ izjavni izraz. Izračunamo lahko resničnostno tabelo tega izraza in na ta način lahko na $f$ gledamo kot na izjavni veznik reda $3$.

\begin{table}[h]
    \centering
    \begin{tabular}{ccc|c}
        $p$ & $q$ & $r$ & $(p \Rightarrow q) \land \lnot r$ \\ \hline
        1 & 1 & 1 & 0 \\
        1 & 1 & 0 & 1 \\
        1 & 0 & 1 & 0 \\
        1 & 0 & 0 & 0 \\
        0 & 1 & 1 & 0 \\
        0 & 1 & 0 & 1 \\
        0 & 0 & 1 & 0 \\
        0 & 0 & 0 & 1
    \end{tabular}
    \caption{Resničnostna tabela izjavnega izraza $(p \Rightarrow q) \land \lnot r$}
\end{table}
\end{zgled}

\section{Tavtologije in enakovredni izrazi}

Če je izjavni izraz \emph{resničen} pri vseh naborih svojih izjavnih spremenljivk, mu rečemo {\definicija tavtologija}. Če je \emph{lažen} pri vseh naborih, mu rečemo {\definicija protislovje}. Če ni niti tavtologija niti protislovje, je {\definicija kontingenten}.

\begin{zgled} \leavevmode
\begin{itemize}
    \item Izjavna izraza $1$ in $p \lor \lnot p$ sta tavtologiji.
    \item Izjavna izraza $0$, $p \land \lnot p$ sta protislovji.
    \item Izračunajmo resničnostne tabele izjavnih izrazov $p \Rightarrow q \Leftrightarrow \lnot p \lor q$, $p \land \lnot (q \Rightarrow p)$ in $p \land (\lnot q \lor p)$. 
    
    \begin{table}[h]
        \centering
        \begin{tabular}{cc|ccc}
            $p$ & $q$ & $p \Rightarrow q \Leftrightarrow \lnot p \lor q$ & $p \land \lnot (q \Rightarrow p)$ & $p \land (\lnot q \lor p)$ \\ \hline
            1 & 1 & 1 & 0 & 1 \\
            1 & 0 & 1 & 0 & 1 \\
            0 & 1 & 1 & 0 & 0 \\
            0 & 0 & 1 & 0 & 0 \\
        \end{tabular}
        \caption{Resničnostne tabele treh izjavnih izrazov}
    \end{table}

    Vidimo, da je prvi izraz tavtologija, drugi protislovje, tretji pa kontingenten.

    \begin{domacanaloga}
        Prepričaj se, da je resničnostna tabela izjavnega izraza $p \land (\lnot q \lor p)$ enaka resničnosti tabeli izjavnega izraza $p$. Sklepaj, da je izjava $p \land (\lnot q \lor p) \Leftrightarrow p$ tavtologija. Na podoben način se prepričaj, da je izraz $p \Rightarrow q \Leftrightarrow \lnot q \Rightarrow \lnot p$ tavtologija.
    \end{domacanaloga}        
\end{itemize}
\end{zgled}

Naj bosta $A$ in $B$ izjavna izraza. Kadar je $A \Leftrightarrow B$ tavtologija, tedaj rečemo, da sta izraza $A$ in $B$ {\definicija enakovredna}. Z drugimi besedami, izraza $A$ in $B$ sta enakovrednosta, kadar imata enaka stolpca v resničnostni tabeli. V tem primeru uporabimo oznako $A \sim B$.

\begin{zgled}
Velja $p \Rightarrow q \sim \lnot p \lor q \sim \lnot q \Rightarrow \lnot p$ in $p \land (\lnot q \lor p) \sim p$.
\end{zgled}

Enakovrednemu paru izjavnih izrazov $A \sim B$ rečemo {\definicija zakoni izjavnega računa}. V vsakem izjavnem izrazu lahko poljuben izraz zamenjamo z enakovrednim in s tem poenostavimo izjavni izraz.

\begin{zgled}
Velja
\[
    p \land (\lnot q \lor p) \Rightarrow (\lnot q \Rightarrow \lnot p) \sim
    p \Rightarrow (p \Rightarrow q).
\]
\end{zgled}

Navedimo nekaj uporabnih zakonov izjavnega računa, ki veljajo za vse izjavne izraze $A,B,C$.
\begin{itemize}
    \item $\lnot 0 \sim 1$, $\lnot 1 \sim 0$
    \item $A \land 0 \sim 0$, $A \land 1 \sim A$, $A \lor 0 \sim A$, $A \lor 1 \sim 1$
    \item $A \land A \sim A$, $A \lor A \sim A$ ({\definicija idempotentnost})
    \item $A \land B \sim B \land A$, $A \lor B \sim B \lor A$ ({\definicija komutativnost})
    \item $A \land (B \land C) \sim (A \land B) \land C$, $A \lor (B \lor C) \sim (A \lor B) \lor C$ ({\definicija asociativnost})
    \item $A \land (A \lor B) \sim A$, $A \lor (A \land B) \sim A$ ({\definicija absorpcija})
    \item $A \land (B \lor C) \sim (A \land B) \lor (A \land C)$, $A \lor (B \land C) \sim (A \lor B) \land (A \lor C)$ ({\definicija distributivnost})
    \item $\lnot \lnot A \sim A$ ({\definicija dvojna negacija})
    \item $\lnot (A \land B) \sim \lnot A \lor \lnot B$, $\lnot (A \lor B) \sim \lnot A \land \lnot B$ ({\definicija De Morganova zakona})
    \item $A \Rightarrow B \sim \lnot B \Rightarrow \lnot A$ ({\definicija kontrapozicija}), $A \Rightarrow B \sim \lnot A \lor B$
    \item $A \Leftrightarrow B \sim (A \Rightarrow B) \land (B \Rightarrow A)$
\end{itemize}

\begin{zgled}
Izjavni izraz iz zadnjega zgleda lahko z naštetimi zakoni poenostavimo do
\[
    p \Rightarrow (p \Rightarrow q) \sim
    \lnot p \lor (\lnot p \lor q) \sim
    (\lnot p \lor \lnot p) \lor q \sim
    \lnot p \lor q \sim
    p \Rightarrow q.
\]
\end{zgled}

\section{DNO in KNO}

Do sedaj smo govorili o tem, kako za dan izjavni izraz določimo njegovo resničnostno tabelo. V tem razdelku si bomo zastavili obratno nalogo. Recimo, da je dana resničnostna tabela nekega kontingentnega izjavnega izraza $A$. Pokazali bomo, da lahko zgolj z uporabo izjavnih veznikov $\lnot$, $\land$, $\lor$ sestavimo izjavni izraz $D$, tako da bo $A \sim D$. Za začetek si oglejmo, kako to naredimo na enem konkretnem primeru.

\begin{zgled}
Naj bo izjavni izraz $A$ dan z resničnostno tabelo $T$.

\begin{table}[h]
    \centering
    \begin{tabular}{ccc|c}
        $p$ & $q$ & $r$ & $A$ \\ \hline
        1 & 1 & 1 & 0 \\
        1 & 1 & 0 & 1 \\
        1 & 0 & 1 & 0 \\
        1 & 0 & 0 & 0 \\
        0 & 1 & 1 & 0 \\
        0 & 1 & 0 & 1 \\
        0 & 0 & 1 & 0 \\
        0 & 0 & 0 & 1
    \end{tabular}
    \caption{Resničnostna tabela $T$ izjavnega izraza $A$}
\end{table}

Iskani izjavni izraz $D$ mora biti resničen natanko v 2., 6. in 8. vrstici tabele $T$. To je res natanko tedaj, ko velja $p \land q \land \lnot r$ ali $\lnot p \land q \land \lnot r$ ali $\lnot p \land \lnot q \land \lnot r$. Torej lahko vzamemo preprosto
\[
    D = (p \land q \land \lnot r) \lor (\lnot p \land q \land \lnot r) \lor (\lnot p \land \lnot q \land \lnot r)
\]
in res velja $D \sim A$. Dobljeni izjavni izraz $D$ lahko še nekoliko poenostavimo. 

\begin{domacanaloga}
    Z uporabo zakonov izjavnega računa se prepričaj, da velja $D \sim \lnot (p \Rightarrow q \Rightarrow r)$.
\end{domacanaloga}
\end{zgled}

Naj bo zdaj $A$ poljuben kontingenten izjavni izraz in $T$ njegova resničnostna tabela. {\definicija Disjunktivna normalna oblika} izraza $A$ je disjunkcija \emph{osnovnih konjunkcij} tistih vrstic, kjer je $A$ resničen. Pri tem je osnovna konjunkcija neke vrstice konjunkcija tistih izjavnih spremenljivk, ki so v tej vrstici resnične, in negacij tistih izjavnih spremenljivk, ki so v tej vrstici lažne. Disjunktivno normalno obliko izraza $A$ krajše pišemo kot $\DNO(A)$.

\begin{zgled}
V zadnjem zgledu je $\DNO(A) = D$.
\end{zgled}

Disjunktivna normalna oblika $\DNO(A)$ je izjavni izraz, ki je zapisan le z uporabo izjavnih veznikov $\lnot$, $\land$, $\lor$. Preverimo še, da je ta izraz res enakovreden začetnemu izrazu $A$.

\begin{trditev}
Naj bo $A$ kontingenten izraz. Potem je $A \sim \DNO(A)$.
\end{trditev}
\begin{dokaz}
Naj bo $T$ resničnostna tabela izraza $A$. Naj bo $i$ poljubna vrstica $T$. Dokazati želimo, da imata $T$ in $\DNO(A)$ v tej vrstici enako resničnostno vrednost.

\begin{description}
    \item[Če je $A$ resničen v vrstici $i$:] Po definiciji disjunktivne normalne oblike je $\DNO(A)$ disjunkcija osnovnih konjunkcij vrstic $T$. V posebnem osnovna konjunkcija vrstice $i$ nastopa v $\DNO(A)$. Ta osnovna konjunkcija je v vrstici $i$ resnična, zato je resnična tudi disjunkcija $\DNO(A)$. \kljuka
    \item[Če je $\DNO(A)$ resničen v vrstici $i$:] V tem primeru mora biti resničen vsaj en člen disjunkcije $\DNO(A)$, torej mora biti v vrstici $i$ resnična osnovna konjunkcija neke vrstice $j$. Osnovna konjunkcija vrstice $j$ je izraz, ki je v vrstici $j$ resničen, v vseh ostalih vrsticah pa lažen. Od tod sledi, da je $i = j$. Torej $\DNO(A)$ vsebuje osnovno konjunkcijo vrstice $i$, zato je $A$ resničen v vrstici $i$. \kljuka
\end{description}

Res imata torej $T$ in $\DNO(A)$ enake resničnostne vrednosti, torej sta $A$ in $\DNO(A)$ enakovredna.
\end{dokaz}

V sestavljanju disjunktivne normalne oblike smo opazovali vrstice, kjer je dan izjavni izraz $A$ resničen. Analogno bi lahko opazovali vrstice, kjer je izraz $A$ lažen, in sestavili izjavni izraz, ki bo \emph{lažen} natanko v vrsticah, v katerih je $A$ lažen. Na primeru pojasnimo, kako lahko na ta način dobimo alternativen izjavni izraz, ki je zopet izražen le z vezniki $\lnot$, $\land$, $\lor$ in je enakovreden izrazu $A$.

\begin{zgled}
Naj bo $T$ resničnostna tabela izjavnega izraza $A$ iz predzadnjega zgleda. Ta tabela je lažna v vrsticah $1,3,4,5,7$. Sestavimo izjavni izraz $K$, ki bo lažen natanko v teh vrsticah. To je enakovredno zahtevi, da je $K$ resničen natanko tedaj, ko nismo v nobeni od vrstic $1,3,4,5,7$. Slednjo zahtevo lahko izrazimo kot konjunkcijo \emph{osnovnih  disjunkcij} vrstic $1,3,4,5,7$, se pravi kot
\[
    K = 
    (\lnot p \lor \lnot q \lor \lnot r) \land
    (\lnot p \lor q \lor \lnot r) \land
    (\lnot p \lor q \lor r) \land
    (o \lor \lnot q \lor \lnot r) \land
    (p \lor q \lor \lnot r).
\]
Tako sestavljenemu izravnemu izrazu $K$ pravimo {\definicija konjunktivna normalna oblika} izraza $A$, krajše $\KNO(A)$. Sorodno kot za disjunktivno normalno obliko se prepričamo, da velja $\KNO(A) \sim A$.
\end{zgled}

\section{Sklepanje v izjavnem računu}

{\definicija Sklep} je končno zaporedje izjav $p_1, p_2, \dots, p_k, z$. Pri tem izjavam $p_1, p_2, \dots, p_k$ pravimo {\definicija predpostavke}, izjavi $z$ pa {\definicija zaključek}.

\begin{zgled}
Opazujmo naslednje zaporedje izjav.

\begin{description}
    \item[$p_1$:] Če je ta žival ptič, potem ima krila.
    \item[$p_2$:] Ta žival nima kril.
    \item[$z$:] Ta žival ni ptič. 
\end{description}

Imamo torej sklep $p_1, p_2, z$. Ta sklep lahko zapišemo nekoliko bolj natančno z upoštevanjem sestavljene strukture izjav v sklepu. Uvedimo izjavi $p$ in $q$ kot:

\begin{description}
    \item[$p$:] Ta žival je ptič.
    \item[$q$:] Ta žival ima krila.  
\end{description}

Sklep $p_1, p_2, z$ lahko torej zapišemo v obliki $p \Rightarrow q, \ \lnot q, \ \lnot p$.
\end{zgled}

Zaporedje izjavnih izrazov $A_1, A_2, \dots, A_k, B$ je {\definicija pravilen sklep} (rekli bomo tudi {\definicija veljaven sklep}), če je zaključek $B$ resničen pri vseh tistih naborih izjavnih spremenljivk, pri katerih so resnične vse predpostavke $A_1, A_2, \dots, A_k$. V tem primeru pišemo $A_1, A_2, \dots, A_k \models B$.

\begin{zgled}
Sklep iz zadnjega zgleda smo zapisali v obliki $p \Rightarrow q, \ \lnot q, \ \lnot p$. Če pri tem gledamo na $p$ in $q$ kot na izjavni spremenljivki (in ne kot oznaki za konkretne izjave), se lahko vprašamo o veljavnosti sklepa. V ta namen sestavimo resničnostno tabelo vseh izjavnih izrazov v sklepu.

\begin{table}[h]
    \centering
    \begin{tabular}{cc|ccc}
        $p$ & $q$ & $p \Rightarrow q$ & $\lnot q$ & $\lnot p$ \\ \hline
        1 & 1 & 1 & 0 & 0 \\
        1 & 0 & 0 & 1 & 0 \\
        0 & 1 & 1 & 0 & 1 \\
        0 & 0 & 1 & 1 & 1
    \end{tabular}
    \caption{Resničnostna tabela sklepa $p \Rightarrow q, \ \lnot q \models \lnot p$}
\end{table}

Edini nabor izjavnih spremenljivk, pri katerih sta resnični obe predpostavki sklepa, je nabor $(p,q)=(0,0)$. Pri tem naboru je resničen tudi zaključek sklepa. Sklep je torej veljaven, se pravi $p \Rightarrow q, \ \lnot q \models \lnot p$.
\end{zgled}

Zabeležimo nekaj pomembnih veljavnih sklepov, ki jim pravimo {\definicija osnovna pravila sklepanja}.

\begin{itemize}
    \item $A, \ A \Rightarrow B \models B$ ({\definicija modus ponens (MP)})
    \item $A \Rightarrow B, \ \lnot B \models \lnot A$ ({\definicija modus tollens (MT)})
    \item $A \lor B, \ \lnot B \models A$ ({\definicija disjunktivni silogizem (DS)})
    \item $A \Rightarrow B, \ B \Rightarrow C \models A \Rightarrow C$ ({\definicija hipotetični silogizem (HS)})
    \item $A \land B \models A$ ({\definicija poenostavitev (Po)})
    \item $A, \ B \models A \land B$ ({\definicija združitev (Zd)})
    \item $A \models A \lor B$ ({\definicija pridružitev (Pr)})
\end{itemize}

Drugo od teh pravil smo spoznali v zadnjem zgledu in se tudi prepričali o veljavnosti. Na podoben način preverimo veljavnosti preostalih.

Pri večjem številu izjavnih spremenljivk je lahko preverjanje veljavnosti sklepa z resničnostno tabelo precej zamudno.

\begin{zgled}
Ali je sklep
\[
    p \Rightarrow q, \ p \lor r, \ q \Rightarrow s, \ r \Rightarrow t, \ \lnot s \models t
\]
veljaven? Ta sklep vsebuje $5$ izjavnih spremenljivk, zato bi za preverjanje veljavnosti morali sestaviti resničnostno tabelo z $2^5 = 32$ vrsticami.
\end{zgled}

V takih primerih si lahko pomagamo z naslednjim alternativnim načinom preverjanja veljavnosti sklepa.

\begin{izrek}[o naravni dedukciji]
Naj bodo $A_1, A_2, \dots, A_k$ izjavni izrazi. Če obstaja zaporedje izjavnih izrazov $B_1, B_2, \dots, B_n$, tako da za vsak $i = 1, 2, \dots, n$ velja vsaj ena od možnosti:
\begin{enumerate}
    \item $B_i$ je eden on $A_1, A_2, \dots, A_k$,
    \item $B_i$ je tavtologija,
    \item $B_i \sim B_j$ za nek $j < i$,
    \item $B_i$ logično sledi iz $B_1, B_2, \dots, B_{i-1}$ po enem od osnovnih pravil sklepanja,
\end{enumerate}
potem velja $A_1, A_2, \dots, A_k \models B$.
\end{izrek}
\begin{dokaz}
Dokazujemo z indukcijo na $n$.

Baza indukcije je $n = 1$. V tem primeru je po predpostavki izjavni izraz $B_1$ lahko le eden od $A_1, A_2, \dots, A_k$ ali tavtologija. V vsakem od teh primerov velja $A_1, A_2, \dots, A_k \models B$. \kljuka

Predpostavimo zdaj, da trditev že velja za $1, 2, \dots, n-1$ in dokažimo veljavnost za $n$. Drži torej $A_1, A_2, \dots, A_k \models B_j$ za vsak $j = 1, 2, \dots, n-1$. Obravnavajmo različne možnosti glede na to, katera od predpostavk velja za $B_n$.
\begin{enumerate}
    \item Če je $B_n$ eden od $A_1, A_2, \dots, A_k$, potem velja $A_1, A_2, \dots, A_k \models B$. \kljuka
    \item Če je $B_n$ tavtologija, potem velja $A_1, A_2, \dots, A_k \models B$. \kljuka
    \item Če je $B_n \sim B_j$ za nek $j < n$, potem po indukcijski predpostavki $A_1, A_2, \dots, A_k \models B_j$ velja $A_1, A_2, \dots, A_k \models B$. \kljuka
    \item Naj nazadnje $B_n$ logično sledi iz $B_1, B_2, \dots, B_{n-1}$. Ker vsak $B_j$ za $j < n$ logično sledi iz $A_1, A_2, \dots, A_k$, velja tudi $A_1, A_2, \dots, A_k \models B$. \kljuka
\end{enumerate}
\end{dokaz}

\begin{zgled}
S pomočjo izreka o naravni dedukciji utemeljimo veljavnost sklepa iz zadnjega zgleda.

\begin{table}[h]
    \centering
    \begin{tabular}{lll}
        $i$ & $B_i$ & utemeljitev \\ \hline
        1 & $p \Rightarrow q$ & predpostavka \\
        2 & $p \lor r$ & predpostavka \\
        3 & $q \Rightarrow s$ & predpostavka \\
        4 & $r \Rightarrow t$ & predpostavka \\
        5 & $\lnot s$ & predpostavka \\
        6 & $p \Rightarrow s$ & HS(1,3) \\
        7 & $\lnot p$ & MT(6,5) \\
        8 & $r \lor p$ & $\sim$ 2 \\
        9 & $r$ & DS(8,7) \\
        10 & \underline{$t$} & MP(9,4) \\
    \end{tabular}
    \caption{Uporaba naravne dedukcije za dokazovanje veljavnosti sklepa}
\end{table}
\end{zgled}

Do zdaj smo se ukvarjali z dokazovanjem veljavnosti danega sklepa. Oglejmo si še, kako pokažemo, da sklep \emph{ni} veljaven. Namesto tega, da sestavimo resničnostno tabelo vseh izjavnih izrazov v sklepu, lahko preprosteje pokažemo le na tisto vrstico resničnoste tabele, ki opazi neveljavnost sklepa. To pomeni, da poiščemo nek nabor vrednosti izjavnih spremenljivk, pri katerem so vse predpostavke resnične, zaključek pa je lažen. Takemu naboru rečemo {\definicija protiprimer}.

\begin{zgled}
Opazujmo naslednji sklep.

    \begin{description}
        \item[$p_1$:] Ta žival ima krila ali pa ni ptič.
        \item[$p_2$:] Če je ta žival pritč, potem leže jajca.
        \item[$p_3$:] Ta žival nima kril.
        \item[$z$:] Torej ta žival ne leže jajc.
    \end{description}

Ta sklep lahko zapišemo nekoliko bolj natančno, če uvedemo izjave $p,q,r$ kot:

\begin{description}
    \item[$p$:] Ta žival ima krila.
    \item[$q$:] Ta žival je ptič.
    \item[$r$:] Ta žival leže jajca.
\end{description}

Sklep lahko torej zapišemo na naslednji način:
\[
    p \lor \lnot q, \ q \Rightarrow r, \ \lnot p \models \lnot r.
\]

Poiščimo protiprimer. Želimo, da so resnične vse predpostavke, zaključek pa lažen. Vzeti moramo torej $r = 1$ in $p = 0$, od koder iz prve predpostavke dobimo še $q = 0$. S to izbiro je tudi druga predpostavka resnična. Nabor $p = 0, q = 0, r = 1$ je torej protiprimer, ki opazi, da sklep ni pravilen.

Če to prevedemo nazaj v človeški jezik, protiprimer torej predstavlja žival, ki nima kril, ni ptič in leže jajca. Konkretna taka žival je na primer kača ali krokodil.
\end{zgled}

Pri dokazovanju bolj kompleksnih sklepov si lahko včasih pomagamo tudi s kakšnimi {\definicija pomožnimi sklepi}. Pogledali si bomo tri take osnovne pomožne sklepe, in sicer pogojni sklep, sklepanje s protislovjem in analiza primerov. Vse te pomožne sklepe bomo izpeljali s pomočjo naslednje trditve.

\begin{trditev}
Velja $A_1, A_2, \dots, A_k \models B$, če in samo če je $A_1 \land A_2 \land \cdot \land A_k \models B$ tavtologija.
\end{trditev}
\begin{dokaz}
Predpostavimo najprej, da velja $A_1, A_2, \dots, A_k \models B$. Če je $A_1 \land A_2 \land \cdots \land A_k$ resnična izjava, potem so resnične vse izjave $A_i$ za $i = 1,2, \dots, k$, torej so vse predpostavke $A_1, A_2, \dots, A_k$ resnične. Od tod sledi, da je resničen tudi zaključek $B$. Torej je $A_1 \land A_2 \land \cdot \land A_k \models B$ tavtologija. \kljuka

Predpostavimo sedaj, da je $A_1 \land A_2 \land \cdot \land A_k \models B$ tavtologija. Dokazati želimo veljavnost sklepa $A_1, A_2, \dots, A_k \models B$. Predpostavimo torej, da so resnične vse predpostavke $A_1, A_2, \dots, A_k$. Potem je resnična tudi njihova konjunkcija $A_1 \land A_2 \land \cdots \land A_k$. Iz predpostavke od tod sledi, da mora biti resnična tudi izjava $B$. Sklep $A_1, A_2, \dots, A_k \models B$ je torej veljaven. \kljuka
\end{dokaz}

{\definicija Pogojni sklep} (PS) uporabimo, kadar dokazujemo sklep, v katerem ima zaključek obliko implikacije. Če želimo sklepati na zaključek $B \Rightarrow C$, dodamo izjavo $B$ med predpostavke in skušamo sklepati na zaključek $C$. Veljavnost tega pomožnega sklepa sledi iz naslednjega izreka.

\begin{izrek}[o pogojnem sklepu]
Velja $A_1, A_2, \dots, A_k \models B \Rightarrow C$, če in samo če velja $A_1, A_2, \dots, A_k, B \models C$.
\end{izrek}
\begin{dokaz}
Naj bo $A = A_1 \land A_2 \land \cdots \land A_k$. Po zadnji trditvi je dovolj dokazati, da je $A \Rightarrow (B \Rightarrow C)$ tavtologija, če in samo če je $A \land B \Rightarrow C$ tavtologija. Ta dva izjavna izraza pa sta si v resnici enakovredna, saj velja
\[
    A \Rightarrow (B \Rightarrow C) \sim \lnot A \lor (\lnot B \lor C) \sim (\lnot A \lor \lnot B) \lor C \sim \lnot (A \land B) \lor C \sim A \land B \Rightarrow C.
\]
Izraza sta torej bodisi oba tavtologiji bodisi nobenen od njiju ni tavtologija. S tem je izrek dokazan.
\end{dokaz}

\begin{zgled}
Dokažimo veljavnost sklepa
\[
    p \Rightarrow q \lor r, \ \lnot r \models p \Rightarrow q.
\]
Ker ima zaključek obliko implikacije, lahko uporabimo pogojni sklep.

\begin{table}[h]
    \centering
    \begin{tabular}{lll}
        $i$ & $B_i$ & utemeljitev \\ \hline
        1 & $p \Rightarrow q \lor r$ & predpostavka \\
        2 & $\lnot r$ & predpostavka \\
        3.1 & $p$ & predpostavka PS \\
        3.2 & $q \lor r$ & MP(3.1, 1) \\
        3.3 & $q$ & DS(3.2, 2) \\
        3 & \underline{$p \Rightarrow q$} & PS(3.1--3.3) \\
    \end{tabular}
    \caption{Uporaba pogojnega sklepa}
\end{table}
\end{zgled}

Pomožni {\definicija sklep s protislovjem} (RA\footnote{Lat. \emph{Reductio ad absurdum}.}) lahko uporabimo kadar koli. Če želimo sklepati na zaključek $B$, dodamo izjavo $\lnot B$ med predpostavke in skušamo sklepati na zaključek $0$. To je še posebej uporabno, kadar ima zaključek obliko negacije $\lnot B$, saj v tem primeru dodatna predpostavka postane $\lnot \lnot B \sim B$. Veljavnost tega pomožnega sklepa sledi iz naslednjega izreka.

\begin{izrek}[o sklepu s protislovjem]
Velja $A_1, A_2, \dots, A_k \models B$, če in samo če velja $A_1, A_2, \dots, A_k, \lnot B \models 0$.
\end{izrek}
\begin{dokaz}
Naj bo $A = A_1 \land A_2 \land \cdots \land A_k$. Kot v dokazu zadnjega izreka je dovolj dokazati, da je $A \Rightarrow B$ tavtologija, če in samo če je $A \land \lnot B \Rightarrow 0$ tavtologija. Ta dva izjavna izraza pa sta si v resnici enakovredna, saj velja
\[
    A \land \lnot B \Rightarrow 0 \sim
    \lnot A \lor \lnot \lnot B \sim
    \lnot A \lor B \sim
    A \Rightarrow B.
\]
\end{dokaz}

\begin{zgled}
Dokažimo veljavnost sklepa
\[
    p \Rightarrow \lnot (p \Rightarrow r), \ q \land s, \ s \models \lnot p.
\]
Ker ima zaključek obliko negacije, še posebej radi uporabimo sklepanje s protislovjem.

\begin{table}[h]
    \centering
    \begin{tabular}{lll}
        $i$ & $B_i$ & utemeljitev \\ \hline
        1 & $p \Rightarrow \lnot(q \Rightarrow r)$ & predpostavka \\
        2 & $q \land s \Rightarrow r$ & predpostavka \\
        3 & $s$ & predpostavka \\
        4.1 & $\lnot \lnot p$ & predpostavka RA \\
        4.2 & $p$ & $\sim$ 4.1 \\
        4.3 & $\lnot (q \Rightarrow r)$ & MP(4.2, 1) \\
        4.4 & $q \land \lnot r$ & $\sim$ 4.3 \\
        4.5 & $q$ & Po(4.4) \\
        4.6 & $q \land s$ & Zd(4.5, 3) \\
        4.7 & $r$ & MP(4.6, 2) \\
        4.8 & $\lnot r \land q$ & $\sim$ 4.4 \\
        4.9 & $\lnot r$ & Po(4.8) \\
        4.10 & $r \land \lnot r$ & Zd(4.7, 4.9) \\
        4.11 & $0$ & $\sim$ 4.10 \\
        4 & \underline{$\lnot p$} & RA(4.1--4.11) \\
    \end{tabular}
    \caption{Uporaba sklepa s protislovjem}
\end{table}
\end{zgled}

Pomožni sklep {\definicija analiza primerov} (AP) uporabimo, kadar dokazujemo sklep, v katerem ima ena od predpostavk obliko disjunkcije $B_1 \lor B_2$. V tem primeru lahko sklep razdelimo na dva preprostejša sklepa, pri čemer prvemu dodamo predpostavko $B_1$, drugemu pa predpostavko $B_2$. Podobno kot pri prejšnjih dveh pomožnih sklepih veljavnost tega pomožnega sklepa sledi iz naslednjega izreka.

\begin{izrek}[o analizi primerov]
Velja $A_1, A_2, \dots, A_k, B_1 \lor B_2 \models C$, če in samo če veljata oba sklepa $A_1, A_2, \dots, A_k, B_1 \models C$ in $A_1, A_2, \dots, A_k, B_2 \models C$.
\end{izrek}

\begin{domacanaloga}
Dokaži izrek o analizi primerov.
\end{domacanaloga}

\begin{zgled}
Dokažimo veljavnost sklepa
\[
    p \Rightarrow r, \ q \Rightarrow r, \ p \lor q \models r.
\]
Tretja predpostavka ima obliko disjunkcije, zato lahko sklep dokažemo s pomočjo analize primerov.

\begin{table}[h]
    \centering
    \begin{tabular}{lll}
        $i$ & $B_i$ & utemeljitev \\ \hline
        1 & $p \Rightarrow r$ & predpostavka \\
        2 & $q \Rightarrow r$ & predpostavka \\
        3 & $p \lor q$ & predpostavka\\
        4.1.1 & $p$ & predpostavka AP(3) \\
        4.1.2 & $r$ & MP(4.1.1, 1) \\
        4.2.1 & $q$ & predpostavka AP(3) \\
        4.2.2 & $r$ & MP(4.1.1, 2) \\
        4. & \underline{$r$} & AP(3, 4.1.1--4.1.2, 4.2.1--4.2.2)
    \end{tabular}
    \caption{Uporaba analize primerov}
\end{table}
\end{zgled}

\chapter{Predikatni račun}

Predikatni račun je \emph{nadgradnja} izjavnega računa, ki omogoča bolj natančno logično izražanje. Najprej si bomo na konkretnih primerih pogledali, katere novosti prinaša predikatni račun. Za tem bomo te formalno definirali in razvili podobno teorijo kot v izjavnem računu.

\section{Motivacija}

Na konkretnih zgledih si oglejmo nekaj simbolov in formul, ki jih bomo videli v predikatnem računu. Ti zgledi nam bodo pomagali, da bomo v naslednjem razdelku lažje predelali formalne definicije predikatnega računa.

\begin{zgled}
Oglejmo si naslednji sklep, zapisan v slovenščini.

\begin{description}
    \item[$p_1$:] Vsak zajec ljubi korenje.
    \item[$p_2$:] Feliks je zajec.
    \item[$z$:] Torej Feliks ljubi korenje.
\end{description}

V izjavnem računu ta sklep zapišemo kot $p_1, p_2 \models z$. Seveda pa ta sklep v izjavnem računu ni veljaven, protiprimer je namreč nabor $p_1 = 1, p_2 = 1, z = 0$. 

Kljub temu se zdi, da je ta sklep v slovenščini vsekakor pravilen. Predikatni račun je nadgradnja izjavnega računa, v katerem izjavne spremeljivke $p_1, p_2, z$ obravnavamo nekoliko podrobneje, tako da bo ta sklep pravilen tudi v tem novem računu.

Uvedimo naslednje oznake:

\begin{description}
    \item[$Z(x)$:] $x$ je zajec.
    \item[$K(x)$:] $x$ ljubi korenje.
    \item[$a$:] Feliks
    \item[$\forall x$:] za vsak $x$
\end{description}

S temi oznakami lahko sklep zapišemo bolj natančno takole:

\begin{description}
    \item[$p_1$:] $\forall x \colon (Z(x) \Rightarrow K(x))$
    \item[$p_2$:] $Z(a)$
    \item[$z$:] $K(a)$
\end{description}

V predikatnem jeziku bomo torej poleg izjavnih vezikov in ločil iz izjavnega računa imeli {\definicija individualne spremenljivke} ($x$), {\definicija individualne konstante} ($a$), {\definicija predikate} ($Z$, $K$) in {\definicija univerzalni kvantifikator} ($\forall$).
\end{zgled}

\begin{zgled}
Prevedimo nekaj izjav iz slovenščine v nov jezik predikatnega računa, kot smo to storili v zadnjem zgledu. Izjave so naslednje:

\begin{enumerate}
    \item Vsi gasilci so hrabri.
    \item Nekateri gasilci so hrabri.
    \item Nekateri gasilci niso hrabri.
    \item Noben gasilec ni hraber.
\end{enumerate}

Uvedimo naslednje oznake:

\begin{description}
    \item[$G(x)$:] $x$ je gasilec.
    \item[$H(x)$:] $x$ je hraber.
    \item[$\exists x$:] obstaja $x$
\end{description}

Izjave v slovenščini lahko s temi oznakami zapišemo na naslednji način:

\begin{enumerate}
    \item $\forall x \colon (G(x) \Rightarrow H(x))$
    \item $\exists x \colon (G(x) \land H(x))$
    \item $\exists x \colon (G(x) \land \lnot H(x))$
    \item $\forall x \colon (G(x) \Rightarrow \lnot H(x))$
\end{enumerate}

Tukaj smo videli še en simbol kvantifikacije, ki ga bomo imeli v predikatnem računu, in sicer {\definicija eksistenčni kvantifikator} ($\exists$).
\end{zgled}

\begin{zgled}
Storimo podobno kot v zadnjih dveh zgledih še za naslednjo matematično izjavo. \emph{Evklidov izrek} nam pove, da obstaja neskončno mnogo praštevil. 

Če želimo to izjavo zapisati na podoben način kot prejšnje izjave v tem poglavju, nam bo v pomoč, če jo prepišemo v nekoliko drugačno obliko, in sicer kot izjavo, da obstajajo poljubno velika praštevila. Še vedno ni jasno, kako bi izrazili del izjave, ki pravi, da obstajajo \emph{poljubno velika} števila, zato bodimo še nekoliko bolj eksplicitni glede tega dela izjave. Evklidov izrek je ekvivalenten izjavi, da za vsako naravno število obstaja večje naravno število, ki je praštevilo. To zadnjo izjavo pa lahko zapišemo v predikatnem računu. 

Uvedimo naslednje oznake:

\begin{description}
    \item[$P(x)$:] $x$ je praštevilo.
    \item[$V(x,y)$:] $x$ je večje od $y$.
\end{description}

V tem zgledu so naše spremenljivke $x$ iz množice naravnih števil, v prejšnjih zgledih pa so bile iz množice živali. Pomembno je, da se na začetku dogovorimo o področju pogovora oziroma {\definicija domeni}, saj izjave, ki jih zapišemo, morda v drugi domeni pomenijo čisto nekaj drugega.

Evklidov izrek lahko nazadnje zapišemo v predikatnem računu na naslednji način:
\[
    \forall x \exists y \colon (V(y,x) \land P(y)).
\]

Tukaj smo srečali predikat $V$, ki je nekoliko drugačen od predikatov, ki smo jih videli do sedaj. Predikat $V$ je namreč \emph{dvomesten} (ima dva parametra), drugi predikati pa so bili vsi \emph{enomestni}.
\end{zgled}

\begin{zgled}
V prejšnjem zgledu smo govorili o praštevilih, za katere smo uvedli predikat $P$. Seveda pa bi lahko samo dejstvo, da je spremenljivka $x$ praštevilo, se pravi da je izjava $P(x)$ resnična, podrobneje opisali v predikatnem računu. To storimo v tem zgledu.

Po definiciji je naravno število $n$ praštevilo, če in samo če ima natanko dva naravna delitelja. Uvedimo naslednje oznake:

\begin{description}
    \item[$f(x,y)$:] produkt $x$ in $y$
    \item[$E(x,y)$:] $x$ je enak $y$.
\end{description}

Definicijo predikata $P$ lahko zapišemo na naslednji način:
\[
    \forall x \colon \left( 
    P(x) \Leftrightarrow V(x,1) \land \forall u \forall v \colon ( E(x, f(u,v)) \Rightarrow E(u,1) \land E(v,1) ).
    \right)
\]

Tukaj smo naleteli še na zadnji fenomen, ki ga bomo imeli v predikatnem računu, in sicer je to {\definicija funkcijski simbol} $f$, ki iz danih števil izračuna njun produkt.
\end{zgled}

\section{Sintaksa predikatnega računa}

Zdaj smo pripravljeni, da formalno definiramo vse koncepte v predikatnem računu. Sistematično bomo našteli sibmole, ki jih uporabljamo, in opisali, kako iz njih sestavljamo izjavne formule.

Naj bo področje pogovora dano z neko množico $D$.

\subsection{Simboli}

V predikatnem računu uporabljamo naslednje simbole:

\begin{enumerate}
    \item {\definicija Individualne konstante}, ki jih bomo označevali z $a,b,c,\dots$. To so \emph{konkretni} elementi množice $D$.
    \item {\definicija Individualne spremenljivke}, ki jih bomo označevali z $x,y,z,\dots$. To so \emph{poljubni} elementi množice $D$.
    \item {\definicija Predikati}, ki jih bomo označevali s $P,Q,R,\dots$. Ti predstavljajo relacije med elementi množice $D$. Predikati so lahko enomestni, dvomestni, \dots, $n$-mestni, \dots
    \item {\definicija Funkcijski simboli}, ki jih bomo označevali s $f,g,h,\dots$. Tudi ti so lahko $n$-mestni za poljubno naravno število $n$. Funkcijski simboli predstavljajo funkcije iz $n$-teric elementov $D$ v $D$.
    \item Izjavni vezniki iz izjavnega računa
    \item {\definicija Simbola kvantifikacije} $\forall$ in $\exists$.
    \item Ločila \texttt{():,}
\end{enumerate}

\subsection{Termi}

Najosnovnejše izjave, ki jih v predikatnem računu sestavljamo z naštetimi simboli, so {\definicija termi}. Definiramo jih induktivno,\footnote{Ta induktivna definicija je podobna kot induktivna definicija izjavnih izrazov v izjavnem računu.} in sicer na naslednji način:

\begin{itemize}
    \item \emph{Osnovni term 1:} Vsaka \emph{individualna konstanta} je term. 
    \item \emph{Osnovni term 2:} Vsaka \emph{individualna spremenljivka} je term.
    \item \emph{Sestavljeni term:} Če je $f$ \emph{$n$-mesten funkcijski simbol} in so $t_1, t_2, \dots, t_n$ termi, potem je $f(t_1,t_2,\dots,t_n)$ term.
\end{itemize}

Termi, ki \emph{ne} vsebujejo nobenih individualnih spremenljivk, se imenujejo {\definicija zaprti termi}.

\begin{zgled}
Naj bo $a$ individualna konstanta, $x,y$ individualni spremenljivki, $f$ enomesten funkcijski simbol, $g$ dvomesten funkcijski simbol. Tedaj so
\[
    x, \ y, \ a, \ f(x), \ f(a), \ f(f(a)), \ g(x,f(f(y)))
\]
termi. Tretji, peti in šesti izmed teh so zaprti termi.
\end{zgled}

\subsection{Izjavne formule}

{\definicija Izjavne formule} so bolj kompleksne izjave v predikatnem računu, ki jih sestavljamo s pomočjo termov. Njihova definicija je induktivna, in sicer:

\begin{itemize}
    \item \emph{Osnovna izjavna formula:} Naj bo $P$ neki $n$-mestni predikat in $t_1, t_2, \dots, t_n$ termi. Potem je $P(t_1, t_2, \dots, t_n)$ izjavna formula. Taki izjavni formuli rečemo {\definicija atomarna}.  
    \item \emph{Sestavljena izjavna formula 1:} Naj bo $F$ neki $n$-mestni izjavni veznik in $\phi_1, \phi_2, \dots, \phi_n$ izjavne formule. Potem je $F(\phi_1, \phi_2, \dots, \phi_n)$ izjavna formula.
    \item \emph{Sestavljena izjavna formula 2:} Naj bo $\phi$ izjavna formula in $x$ individualna spremenljivka. Potem sta $(\forall x \colon \phi)$ in $(\exists x \colon \phi)$ izjavni formuli. Pri tem je $\forall x$ {\definicija univerzalni kvantifikator}, $\exists x$ {\definicija eksistenčni kvantifikator}, formula $\phi$ pa je {\definicija doseg kvantifikatorja}.
\end{itemize}

Kot pri izjavnem računu sprejmemo {\definicija dogovor o prednostnem vrstnem redu kvantifikatorjev in opuščjanju ločil}:
\begin{itemize}
    \item za izjavne veznike in zunanje oklepaje velja dogovor iz izjavnega računa,
    \item kvantifikatorji imajo prednost pred izjavnimi vezniki,
    \item ločila med zaporednimi kvantifikatorji izpuščamo.
\end{itemize}

\begin{zgled}
V zgledih na začetku tega poglavja smo srečali naslednje izjavne formule:
\[
    Z(x), \ K(x), \ Z(a), \ K(a), \ \lnot H(x), \ \forall x \colon (G(x) \Rightarrow H(x)), \ \forall x \exists y \colon (V(y,x) \land P(x)).
\]
\end{zgled}

Analogno definiciji zaprtosti termov uvedemo koncept zaprtosti izjavnih formul. Nastop individualne spremenljivke v izjavni formuli je {\definicija vezan}, če je del kvantifikatorja ali leži v dosegu kvantifikatorja, ki vsebuje to spremenljivko. Če nastop ni vezan, je {\definicija prost}. Izjavna formula je {\definicija zaprta}, če so vsi nastopi individualnih spremenljivk v njej vezani.

\begin{zgled}
Naslednje izjavne formule so zaprte:
\[
    P(a), \ \forall x \colon P(x), \ \forall x \exists y \colon V(x,y).
\]
Naslednje izjavne formule pa niso zaprte:
\[
    P(x), \ \exists y \colon V(x,y), \ \forall x \colon (V(x,y) \Rightarrow \exists P(y)).
\]
\end{zgled}

Postojmo le še pri enem osnovnem konceptu v zvezi z izjavnimi formulami, in sicer substitucijo. Naj bo $\phi$ izjavna formula z individualno spremenljivko $x$, tako da lahko zapišemo formulo kot $\phi(x)$. Če je $t$ nek term, potem s $\phi(t)$ označimo izjavno formulo, ki jo dobimo iz $\phi(x)$, če v njej vse \emph{proste} nastope $x$ zamenjamo s $t$. V tem primeru rečemo, da smo $\phi(t)$ dobili s {\definicija substitucijo} iz $\phi(x)$ spremenljivke $x$ s termom $t$.

\begin{zgled}
Naj bo $\phi(x) = \exists y \colon R(x,y)$ in $t = f(a)$. Potem je $\phi(t) = \exists y \colon R(f(a), y)$.
\end{zgled}

Analogno lahko v formuli $\phi(x_1, x_2, \dots, x_n)$ z različnimi individualnimi spremenljivkami $x_1, x_2, \dots, x_n$ izvedemo substitucijo vseh prostih nastopov $x_i$ s termom $t_i$ za $i = 1,2,\dots,n$.

\chapter{Relacije}
\chapter{Urejenosti}
\chapter{Grafi}

\end{document}
